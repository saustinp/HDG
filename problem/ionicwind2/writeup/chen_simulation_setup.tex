\documentclass[12pt, a4paper]{report}
\usepackage[utf8]{inputenc}
\usepackage{graphicx}
\usepackage{caption}
\usepackage{palatino}
\usepackage{float}
\usepackage{subcaption}
% \usepackage{siunitx}
\usepackage{bm}
\setlength{\topmargin}{-0.5in}
\setlength{\topmargin}{0in}
\setlength{\headheight}{0in}
\setlength{\topsep}{0in}
\setlength{\textheight}{9in}
\setlength{\oddsidemargin}{0in}
\setlength{\evensidemargin}{0in}
\setlength{\textwidth}{6.5in}
\renewcommand{\thesection}{\arabic{section}}
\usepackage{gensymb}
\usepackage{amssymb}
\usepackage{hyperref}
% \usepackage{physics}
\usepackage{amsmath,mleftright}
% \usepackage{unicode-math} % also loads fontspec
\hypersetup{
    colorlinks=true,
    linkcolor=blue,
    filecolor=magenta,      
    urlcolor=cyan,
}

\usepackage{xcolor}
\definecolor{light-gray}{gray}{0.95}
\newcommand{\code}[1]{\colorbox{light-gray}{\texttt{#1}}}


\usepackage{bm}

\newcommand{\bv}{{\bm v}}
\newcommand{\bfo}{{\bm f}}
\newcommand{\bq}{{\bm q}}
\newcommand{\br}{{\bm r}}
\newcommand{\bPi}{{\bm \Pi}}
\newcommand{\bOmega}{{\bm \Omega}}
\newcommand{\be}{{\bm e}}

\newcommand{\ur}{{u_r}}
\newcommand{\ut}{{u_\theta}}
\newcommand{\uz}{{u_z}}



% For code block insertion
\usepackage{listings}
\usepackage{color}

\definecolor{dkgreen}{rgb}{0,0.6,0}
\definecolor{gray}{rgb}{0.5,0.5,0.5}
\definecolor{mauve}{rgb}{0.58,0,0.82}

\lstset{frame=tb,
  language=Matlab,
  aboveskip=3mm,
  belowskip=3mm,
  showstringspaces=false,
  columns=flexible,
  basicstyle={\small\ttfamily},
  numbers=none,
  numberstyle=\tiny\color{gray},
  keywordstyle=\color{blue},
  commentstyle=\color{dkgreen},
  stringstyle=\color{mauve},
  breaklines=true,
  breakatwhitespace=true,
  tabsize=3
}


\begin{document}

\title{Ionic Wind Simulation Notes}
\date{\today}
\maketitle

\noindent

\section{Objectives}
The objective of this research is to replicate the results of Chen et al. (2017), "A Self-Consistent Model of Ionic Wind Generation by Negative Corona Discharges in Air With Experimental Validation".

\section{Governing Equations}
The following equations constitute equations 1-5 in Chen et al. They will be solved simultaneously in Exasim using the Convection-Diffusion transport model (Model D). The newly-introduced DAE "subproblem" module will be used to separately solve the species transport/diffusion problem and the electrostatic problem.

\begin{equation}
    \frac{\partial n_e}{\partial t} + \nabla \cdot (-\mu_e\vec{E}n_e - D_e\nabla n_e) = \alpha n_e|\mu_e\vec{E}| -\eta n_e|\mu_e\vec{E}| - k_{ep}n_en_p
\end{equation}

\begin{equation}
    \frac{\partial n_p}{\partial t} + \nabla \cdot (\mu_p\vec{E}n_p - D_p\nabla n_p) = \alpha n_e|\mu_e\vec{E}| - k_{np}n_nn_p - k_{ep}n_en_p
\end{equation}

\begin{equation}
    \frac{\partial n_n}{\partial t} + \nabla \cdot (-\mu_n\vec{E}n_n - D_n\nabla n_n) = \eta n_e|\mu_e\vec{E}| - k_{np}n_nn_p
\end{equation}

\begin{equation}
    \nabla^2\Phi = -\frac{e(n_p-n_e-n_n)}{\epsilon}
\end{equation}

\begin{equation}
    \vec{E} = -\nabla\Phi
\end{equation}

Additionally, the one-way coupling equations from the EHD model to the gas dynamics model (N-S) is provided in Eqns. 6-8:

\begin{equation}
    f_{ehd} = e(n_p -n_e-n_n) \vec{E}
\end{equation}

\begin{equation}
    \nabla \cdot \vec{u} = 0
\end{equation}

\begin{equation}
    \rho_g \left( \frac{\partial \vec{u}}{\partial t} + \vec{u}\cdot\nabla\vec{u}  \right) = -\nabla p + \mu_v\nabla^2\vec{u} + f_{ehd}
\end{equation}

\section{Cylindrical Coordinates}

The gradient and divergence operators in cylindrical coordinates $(r,\theta,z)$ are given, for a scalar function $f$ and a vector field ${\bm f}$, by

\[
\begin{array}{rcl}
\nabla f  & = & \displaystyle \frac{\partial f}{\partial r} \be_r +  \frac{1}{r} \frac{\partial f}{\partial \theta} \be_{\theta} + \frac{\partial f}{\partial z} \be_z \\[2ex]
\nabla \cdot {\bm f}   & = & \displaystyle \frac{1}{r} \frac{\partial  (r f_r)}{\partial r}  +  \frac{1}{r}  \frac{\partial    f_\theta}{\partial \theta}  +  \frac{\partial f_z}{\partial z} \\[2ex]
\end{array}
\]

\noindent
If we assume axial symmetry all variable are function of $(r,z)$ only.

\section{ Fully Conservative Form}

Assuming axial symmetry, the above equations can be written as 

\begin{center}
\[
{\bm M} \frac{\partial {\bf U}}{\partial t} +  \frac{1}{r} \frac{\partial (r {\bf F}_r)}{\partial r} +  \frac{\partial {\bf F}_z}{\partial z} -  {\bf S}  = {\bf 0}
\]
\end{center}
Were
\begin{equation} 
{\bm U} = \left( \begin{array}{c} n_e \\ n_p \\ n_n \\ \Phi \end{array} \right),  \quad {\bm F}_r = \left( \begin{array}{c} - \mu_e n_e E_r - D_e (q_e)_r \\ \mu_p n_p E_r - D_p (q_p)_r \\ -\mu_n n_n E_r - D_n (q_n)_r \\ -E_r \end{array} \right),  \quad {\bm F}_z = \left( \begin{array}{c} - \mu_e n_e E_z - D_e (q_e)_z \\ -\mu_p n_p E_z - D_p (q_p)_z \\ -\mu_n n_n E_z - D_n (q_n)_z \\ -E_z \end{array} \right),
\end{equation}
\begin{equation} 
{\bm S} = \left( \begin{array}{c} \alpha n_e|\mu_e\vec{E}| -\eta n_e|\mu_e\vec{E}| - k_{ep}n_en_p \\ \alpha n_e|\mu_e\vec{E}| - k_{np}n_nn_p - k_{ep}n_en_p \\ \eta n_e|\mu_e\vec{E}| - k_{np}n_nn_p \\ -\frac{e(n_p-n_e-n_n)}{\epsilon} \end{array} \right),  \quad {\bm M} = \left( \begin{array}{cccc} 
1 & 0 & 0 & 0 \\
0 & 1 & 0 & 0 \\
0 & 0 & 1 & 0 \\
0 & 0 & 0 & 0 \end{array} \right)
\end{equation}
and
\begin{equation} 
{\bm Q} = \left( \begin{array}{cc} (q_e)_r & (q_e)_z \\ (q_p)_r & (q_p)_z \\ (q_n)_r & (q_n)_z \\ -E_r & -E_z \end{array} \right),  
\end{equation}
with 
\begin{equation} 
{\bm Q} = \nabla {\bm U}.
\end{equation}
Note that under the  axisymmetry assumption, the gradient operators in cylindrical and cartesian coordinates are the same.


\section{Weak form and the Divergence Theorem}
The elemental volume becomes $dV = r dr dz$ and for any ${\bf W}$ we can write the following weighted residual form



\begin{center}
\[\begin{array}{c} \displaystyle
\int_V \left({\bm M} \frac{\partial {\bf U}}{\partial t} +  \frac{1}{r} \frac{\partial (r {\bf F}_r)}{\partial r}  + \frac{\partial {\bf F}_z}{\partial z} +   {\bf S}\right) {\bf W} \ r dr dz  = 0 ,
\end{array} \]
\end{center}
or, integrating by parts, 
\begin{center}
\[\begin{array}{c}
\displaystyle \int_V \left(({\bm M} r) \frac{\partial {\bf U}}{\partial t}   +r{\bf S}\right) {\bf W} \ dr  dz  
\displaystyle + \int_S\left(  r {\bf F}_r {\bm n}_r +  r {\bf F}_z {\bm n}_z \right) {\bf W}  \, dS \\[4ex] - \displaystyle \int_V \left(  r {\bf F}_r\frac{\partial {\bf W}}{\partial r}  + r {\bf F}_z\frac{\partial {\bf W}}{\partial z} \right) \ dr  dz  = {\bf 0}
\end{array}
\]
\end{center}

Note that in the integrals in the second and third lines the {\em effective} $dV$ becomes $dr dz$. The only differences between cylindrical and cartesian coordinates thus

\begin{itemize}
\item Multiply $\bm M$ and $\bm S$ by $r$. Note that this will require a modified mass matrix.
\item Multiply ${\bm F}_r$ and ${\bm F}_z$ by $r$

\end{itemize}

\section{Nondimensionalization}
\noindent
Because of the wide range of scales used, it is important to nondimensionalize the problem to prevent numerical instability. The solution variables as well as spatial and temporal variables were nondimensionalized.

\subsection{Nondimensional groups}
The following nondimensional groups were chosen:

\begin{itemize}
    \item $n_{ref} = \frac{\epsilon_0 E_{bd}}{e r_{tip}}$
    \item $E_{ref} = E_{bd}$
    \item $\Phi_{ref} = E_{bd}r_{tip}$
    \item $t_{ref} = \frac{r_{tip}}{\mu_{e,ref} E_{bd}}$
    \item $l_{ref} = r_{tip}$
\end{itemize}

Where $\mu_{e,ref}$ is taken at the reduced electric field value of $\frac{E_{bd}}{N}$, where $N$, the neutral number density, is computed using the ideal gas equation of state at standard temperature and pressure:

\begin{itemize}
    \item $pV = Nk_BT$
    \item $P = 101325$ $Pa$
    \item $V=1$ $m^3$
    \item $T = 273.15$ $K$
    \item $k_B = 1.380649\times 10 ^{-23}$ $\frac{m^2kg}{s^{2} K}$
    \item $\implies N = 2.6868 \times 10 ^{25}$ $\frac{particles}{m^3}$
\end{itemize}

\noindent
Thus, the independent and dependent variables may be expressed as the following. The asterisk indicates a nondimensional quantity.

\begin{itemize}
    \item $n_e = \frac{n^*_e \epsilon_0 E_{bd}}{e r_{tip}}$
    \item $n_p = \frac{n^*_p \epsilon_0 E_{bd}}{e r_{tip}}$
    \item $n_n = \frac{n^*_n \epsilon_0 E_{bd}}{e r_{tip}}$
    \item $\vec{E} = \vec{E^*}E_{bd}$
    \item $\Phi = \Phi^* E_{bd} r_{tip}$
    \item $t = \frac{t^*r_{tip}}{\mu_{e,ref} E_{bd}}$
    \item $r = r^*r_{tip}$
    \item $z = z^*r_{tip}$
\end{itemize}

\noindent
Where $r_{tip}$ is the needle tip radius of curvature, 220$\mu m$, and $E_{bd}$ is the breakdown electric field strength in air, $3\times10^6 \frac{V}{m}$.\\

\noindent
We can re-write the governing equations (section 2) using the nondimensional groups, taking care to nondimensionalize the partial derivatives with respect to space and time as well:

\begin{align*}
        \frac{\partial (n_e^* n_{ref})}{\partial \left(\frac{t^*r_{tip}}{\mu_e E_{bd}}  \right)} + \nabla \cdot\frac{\left( -\mu_e\vec{E^*}E_{bd}(n_e^*n_{ref}) - D_e\nabla \left(\frac{n_e^* n_{ref}}{r_{tip}}  \right) \right)}{r_{tip}} = \\ (\alpha - \eta) (n_e^*n_{ref})|\mu_e(\vec{E^*}E_{bd})| - k_{ep}n_e^*n_p^*n_{ref}^2
\end{align*}

\begin{align*}
    \frac{\partial (n_p^* n_{ref})}{\partial \left(\frac{t^*r_{tip}}{\mu_e E_{bd}}  \right)} + \nabla \cdot\frac{\left(\mu_p\vec{E^*}E_{bd}(n_p^*n_{ref}) - D_p\nabla \left(\frac{n_p^* n_{ref}}{r_{tip}}  \right) \right)}{r_{tip}} = \\ \alpha n_e^*n_{ref} |\mu_e(\vec{E^*}E_{bd})| - n_p^*n_{ref}^2\left(k_{np}n_n^* + k_{ep}n_e^*\right)
\end{align*}

\begin{align*}
    \frac{\partial (n_n^* n_{ref})}{\partial \left(\frac{t^*r_{tip}}{\mu_e E_{bd}}  \right)} + \nabla \cdot\frac{\left(-\mu_n\vec{E^*}E_{bd}(n_n^*n_{ref}) - D_n\nabla \left(\frac{n_n^* n_{ref}}{r_{tip}}  \right) \right)}{r_{tip}} = \\ \eta n_e^*n_{ref}|\mu_e(\vec{E^*}E_{bd})| - k_{np}n_n^*n_p^*n_{ref}^2
\end{align*}

\begin{align*}
    \nabla^2 \left(\frac{\Phi^*E_{bd}r_{tip}}{r_{tip}^2}\right) = -\frac{\epsilon_0 e E_{bd}}{\epsilon_0 e r_{tip}} (n_p^*-n_e^*-n_n^*)
\end{align*}

\noindent
Simplifying:
\begin{align*}
        \frac{\partial n_e^* }{\partial t^*} + \nabla \cdot \left(- \frac{\mu_e}{\mu_{e,ref}} \vec{E^*}n_e^* - \frac{{D_e}}{\mu_{e,ref} E_{bd}r_{tip}}\nabla (n_e^*)\right) = (\alpha - \eta)\frac{\mu_e}{\mu_{e,ref}}r_{tip} n_e^*|\vec{E^*}| - \frac{k_{ep} \epsilon_0}{e \mu_{e,ref}} n_e^*n_p^*
\end{align*}

\begin{align*}
        \frac{\partial n_p^* }{\partial t^*} + \nabla \cdot \left(\frac{\mu_p}{\mu_{e,ref}} \vec{E^*}n_p^* - \frac{{D_p}}{\mu_{e,ref} E_{bd}r_{tip}}\nabla (n_p^*)\right) = \alpha \frac{\mu_e}{\mu_{e,ref}}r_{tip} n_e^*|\vec{E^*}| - \left(\frac{\epsilon_0 n_p^*}{e \mu_{e,ref}} \right) \left(k_{np}n_n^* + k_{ep}n_e^*\right)
\end{align*}

\begin{align*}
        \frac{\partial n_n^* }{\partial t^*} + \nabla \cdot \left(-\frac{\mu_n}{\mu_{e,ref}} \vec{E^*}n_n^* - \frac{{D_n}}{\mu_{e,ref} E_{bd}r_{tip}}\nabla (n_n^*)\right) = \eta \frac{\mu_e}{\mu_{e,ref}}r_{tip} n_e^*|\vec{E^*}| - \left(\frac{k_{np} \epsilon_0}{e \mu_{e,ref}} \right) n_n^*n_p^*
\end{align*}

\begin{align*}
    \nabla^2\Phi^* = n_e^*+n_n^*-n_p^*
\end{align*}

Then don't forget to multiply by r on both the source terms and the fluxes -> the r coordinate is nondimensional so it doesn't matter.

\section{\bf \large Boundary Conditions}
Note: Boundary numbering follows the boundary numbering in the paper
\subsection{Boundary 1}

\begin{table}[!h]
    \centering\begin{tabular}{c|c|c}
        Equation & Boundary condition & Boundary condition type\\ \hline
        1 & Total flux $-\vec{n} \cdot\left(-\mu_{\mathrm{e}} \vec{E} -D_{\mathrm{e}} \nabla n_{\mathrm{e}}\right) =\gamma n_{\mathrm{p}}|\mu_{\mathrm{p}} \vec{E}|$ & Neumann \\ \hline
        2 & Outflow, $\vec{n} \cdot\left(-D_{\mathrm{p}} \nabla n_{\mathrm{p}}\right)=0$  & Neumann \\ \hline
        3 & $n_n=0$ & Dirichlet\\ \hline
        4 & $\Phi=-U_a$ & Dirichlet \\ \hline

    \end{tabular}
    \caption{Boundary conditions for the emitter tip (Boundary surface 1)}
\end{table}
\clearpage

\subsection{Boundary 2}
\begin{table}[!h]
    \centering\begin{tabular}{c|c|c}
        Equation & Boundary condition & Boundary condition type\\ \hline
        1 & Axial symmetry $\frac{\partial n_{\mathrm{e}}}{\partial r}=0$ &  Neumann\\ \hline
        2 & Axial symmetry $\frac{\partial n_{\mathrm{p}}}{\partial r}=0$  & Neumann \\ \hline
        3 & Axial symmetry $\frac{\partial n_{\mathrm{n}}}{\partial r}=0$ & Neumann\\ \hline
        4 & Axial symmetry $\frac{\partial \phi}{\partial r}=0 \quad$ & Neumann \\ \hline

    \end{tabular}
    \caption{Boundary conditions for  (Boundary surface 2)}
\end{table}

\subsection{Boundary 3}

\begin{table}[!h]
    \centering\begin{tabular}{c|c|c}
        Equation & Boundary condition & Boundary condition type\\ \hline
        1 & Open boundary $\begin{array}{c}\vec{n} \cdot\left(-D_{\mathrm{e}} \nabla n_{\mathrm{e}}\right)=0 ; \vec{n} \cdot\left(-\mu_{\mathrm{e}} \vec{E}\right) \geqslant 0 \\ n_{\mathrm{e}}=0 ; \vec{n} \cdot\left(-\mu_{\mathrm{e}} \vec{E}\right)<0\end{array} $ & Neumann/Dirichlet \\ \hline
        2 & Open boundary $\begin{array}{c}\vec{n} \cdot\left(-D_{\mathrm{p}} \nabla n_{\mathrm{p}}\right)=0 ; \vec{n} \cdot\left(-\mu_{\mathrm{p}} \vec{E}\right) \geqslant 0 \\ n_{\mathrm{p}}=0 ; \vec{n} \cdot\left(-\mu_{\mathrm{p}} \vec{E}\right)<0\end{array} $   & Neumann/Dirichlet \\ \hline
        3 & Open boundary $\begin{array}{c}\vec{n} \cdot\left(-D_{\mathrm{n}} \nabla n_{\mathrm{n}}\right)=0 ; \vec{n} \cdot\left(-\mu_{\mathrm{n}} \vec{E}\right) \geqslant 0 \\ n_{\mathrm{n}}=0 ; \vec{n} \cdot\left(-\mu_{\mathrm{n}} \vec{E}\right)<0\end{array} $  & Neumann/Dirichlet\\ \hline
        4 & Ground $\phi=0$ & Dirichlet \\ \hline

    \end{tabular}
    \caption{Boundary conditions for  (Boundary surface 3)}
\end{table}

\subsection{Boundary 4}

\begin{table}[!h]
    \centering\begin{tabular}{c|c|c}
        Equation & Boundary condition & Boundary condition type\\ \hline
        1 & Outflow $\vec{n} \cdot\left(-D_{\mathrm{e}} \nabla n_{\mathrm{e}}\right)=0\quad$ & Neumann \\ \hline
        2 & $n_p=0$  & Dirichlet \\ \hline
        3 & Outflow $\vec{n} \cdot\left(-D_{\mathrm{n}} \nabla n_{\mathrm{n}}\right)=0$ &Neumann \\ \hline
        4 & Ground $\phi=0$ & Dirichlet \\ \hline

    \end{tabular}
    \caption{Boundary conditions for  (Boundary surface 4)}
\end{table}
\clearpage

\subsection{Boundary 5 and 6}

\begin{table}[!h]
    \centering\begin{tabular}{c|c|c}
        Equation & Boundary condition & Boundary condition type\\ \hline
        1 & Open boundary $\begin{array}{c}\vec{n} \cdot\left(-D_{\mathrm{e}} \nabla n_{\mathrm{e}}\right)=0 ; \vec{n} \cdot\left(-\mu_{\mathrm{e}} \vec{E}\right) \geqslant 0 \\ n_{\mathrm{e}}=0 ; \vec{n} \cdot\left(-\mu_{\mathrm{e}} \vec{E}\right)<0\end{array} $ & Neumann/Dirichlet \\ \hline
        2 & Open boundary $\begin{array}{c}\vec{n} \cdot\left(-D_{\mathrm{p}} \nabla n_{\mathrm{p}}\right)=0 ; \vec{n} \cdot\left(-\mu_{\mathrm{p}} \vec{E}\right) \geqslant 0 \\ n_{\mathrm{p}}=0 ; \vec{n} \cdot\left(-\mu_{\mathrm{p}} \vec{E}\right)<0\end{array} $   & Neumann/Dirichlet \\ \hline
        3 & Open boundary $\begin{array}{c}\vec{n} \cdot\left(-D_{\mathrm{n}} \nabla n_{\mathrm{n}}\right)=0 ; \vec{n} \cdot\left(-\mu_{\mathrm{n}} \vec{E}\right) \geqslant 0 \\ n_{\mathrm{n}}=0 ; \vec{n} \cdot\left(-\mu_{\mathrm{n}} \vec{E}\right)<0\end{array} $  & Neumann/Dirichlet\\ \hline
        4 & Zero charge $\vec{n} \cdot\left(\epsilon \vec{E}\right)<0$ & Neumann \\ \hline

    \end{tabular}
    \caption{Boundary conditions for  (Boundary surfaces 5 and 6)}
\end{table}

\section{\bf \large Code Review}



\begin{figure}[!h]
    \centering
    \includegraphics*[width=.8\linewidth]{indicator_fcn.png}
    \caption{Indicator function for $E \cdot n$ used in this problem}
\end{figure}

\begin{figure}[!h]
    \centering
    \includegraphics*[width=.99\linewidth]{exasim_bc_index.png}
    \caption{Computational domain of Chen 2017 with Exasim BC indices overlaid}
    \label{<label>}
\end{figure}



\end{document}